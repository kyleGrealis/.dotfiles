% Options for packages loaded elsewhere
% Options for packages loaded elsewhere
\PassOptionsToPackage{unicode}{hyperref}
\PassOptionsToPackage{hyphens}{url}
\PassOptionsToPackage{dvipsnames,svgnames,x11names}{xcolor}
%
\documentclass[
  number,
  preprint,
  3p,
  onecolumn]{elsarticle}
\usepackage{xcolor}
\usepackage{amsmath,amssymb}
\setcounter{secnumdepth}{5}
\usepackage{iftex}
\ifPDFTeX
  \usepackage[T1]{fontenc}
  \usepackage[utf8]{inputenc}
  \usepackage{textcomp} % provide euro and other symbols
\else % if luatex or xetex
  \usepackage{unicode-math} % this also loads fontspec
  \defaultfontfeatures{Scale=MatchLowercase}
  \defaultfontfeatures[\rmfamily]{Ligatures=TeX,Scale=1}
\fi
\usepackage{lmodern}
\ifPDFTeX\else
  % xetex/luatex font selection
\fi
% Use upquote if available, for straight quotes in verbatim environments
\IfFileExists{upquote.sty}{\usepackage{upquote}}{}
\IfFileExists{microtype.sty}{% use microtype if available
  \usepackage[]{microtype}
  \UseMicrotypeSet[protrusion]{basicmath} % disable protrusion for tt fonts
}{}
\makeatletter
\@ifundefined{KOMAClassName}{% if non-KOMA class
  \IfFileExists{parskip.sty}{%
    \usepackage{parskip}
  }{% else
    \setlength{\parindent}{0pt}
    \setlength{\parskip}{6pt plus 2pt minus 1pt}}
}{% if KOMA class
  \KOMAoptions{parskip=half}}
\makeatother
% Make \paragraph and \subparagraph free-standing
\makeatletter
\ifx\paragraph\undefined\else
  \let\oldparagraph\paragraph
  \renewcommand{\paragraph}{
    \@ifstar
      \xxxParagraphStar
      \xxxParagraphNoStar
  }
  \newcommand{\xxxParagraphStar}[1]{\oldparagraph*{#1}\mbox{}}
  \newcommand{\xxxParagraphNoStar}[1]{\oldparagraph{#1}\mbox{}}
\fi
\ifx\subparagraph\undefined\else
  \let\oldsubparagraph\subparagraph
  \renewcommand{\subparagraph}{
    \@ifstar
      \xxxSubParagraphStar
      \xxxSubParagraphNoStar
  }
  \newcommand{\xxxSubParagraphStar}[1]{\oldsubparagraph*{#1}\mbox{}}
  \newcommand{\xxxSubParagraphNoStar}[1]{\oldsubparagraph{#1}\mbox{}}
\fi
\makeatother


\usepackage{longtable,booktabs,array}
\usepackage{calc} % for calculating minipage widths
% Correct order of tables after \paragraph or \subparagraph
\usepackage{etoolbox}
\makeatletter
\patchcmd\longtable{\par}{\if@noskipsec\mbox{}\fi\par}{}{}
\makeatother
% Allow footnotes in longtable head/foot
\IfFileExists{footnotehyper.sty}{\usepackage{footnotehyper}}{\usepackage{footnote}}
\makesavenoteenv{longtable}
\usepackage{graphicx}
\makeatletter
\newsavebox\pandoc@box
\newcommand*\pandocbounded[1]{% scales image to fit in text height/width
  \sbox\pandoc@box{#1}%
  \Gscale@div\@tempa{\textheight}{\dimexpr\ht\pandoc@box+\dp\pandoc@box\relax}%
  \Gscale@div\@tempb{\linewidth}{\wd\pandoc@box}%
  \ifdim\@tempb\p@<\@tempa\p@\let\@tempa\@tempb\fi% select the smaller of both
  \ifdim\@tempa\p@<\p@\scalebox{\@tempa}{\usebox\pandoc@box}%
  \else\usebox{\pandoc@box}%
  \fi%
}
% Set default figure placement to htbp
\def\fps@figure{htbp}
\makeatother





\setlength{\emergencystretch}{3em} % prevent overfull lines

\providecommand{\tightlist}{%
  \setlength{\itemsep}{0pt}\setlength{\parskip}{0pt}}



 
\usepackage[]{natbib}
\bibliographystyle{elsarticle-num}


\usepackage{booktabs}
\usepackage{longtable}
\usepackage{array}
\usepackage{multirow}
\usepackage{wrapfig}
\usepackage{float}
\usepackage{colortbl}
\usepackage{pdflscape}
\usepackage{tabu}
\usepackage{threeparttable}
\usepackage{threeparttablex}
\usepackage[normalem]{ulem}
\usepackage{makecell}
\usepackage{xcolor}
\usepackage{caption}
\usepackage{anyfontsize}
\usepackage{lscape}
\usepackage{pdflscape}
\usepackage{rotating}
\makeatletter
\@ifpackageloaded{caption}{}{\usepackage{caption}}
\AtBeginDocument{%
\ifdefined\contentsname
  \renewcommand*\contentsname{Table of contents}
\else
  \newcommand\contentsname{Table of contents}
\fi
\ifdefined\listfigurename
  \renewcommand*\listfigurename{List of Figures}
\else
  \newcommand\listfigurename{List of Figures}
\fi
\ifdefined\listtablename
  \renewcommand*\listtablename{List of Tables}
\else
  \newcommand\listtablename{List of Tables}
\fi
\ifdefined\figurename
  \renewcommand*\figurename{Figure}
\else
  \newcommand\figurename{Figure}
\fi
\ifdefined\tablename
  \renewcommand*\tablename{Table}
\else
  \newcommand\tablename{Table}
\fi
}
\@ifpackageloaded{float}{}{\usepackage{float}}
\floatstyle{ruled}
\@ifundefined{c@chapter}{\newfloat{codelisting}{h}{lop}}{\newfloat{codelisting}{h}{lop}[chapter]}
\floatname{codelisting}{Listing}
\newcommand*\listoflistings{\listof{codelisting}{List of Listings}}
\makeatother
\makeatletter
\usepackage{pdflscape}
\makeatother
\makeatletter
\makeatother
\makeatletter
\@ifpackageloaded{caption}{}{\usepackage{caption}}
\@ifpackageloaded{subcaption}{}{\usepackage{subcaption}}
\makeatother
\journal{Journal of Substance Use and Addiction Treatment}
\usepackage{bookmark}
\IfFileExists{xurl.sty}{\usepackage{xurl}}{} % add URL line breaks if available
\urlstyle{same}
\hypersetup{
  pdftitle={Applying Machine Learning in Predicting Medication Treatment Outcomes for Opioid Use Disorder},
  pdfauthor={Raymond R. Balise; Kyle Grealis; Laura Brandt; Sean X. Luo; Gabriel Odom; Guertson Jean-Baptiste; Daniel J. Feaster},
  pdfkeywords={machine learning, opioid use disorder, treatment
outcomes, medication treatment, prediction modeling},
  colorlinks=true,
  linkcolor={blue},
  filecolor={Maroon},
  citecolor={Blue},
  urlcolor={Blue},
  pdfcreator={LaTeX via pandoc}}


\setlength{\parindent}{6pt}
\begin{document}

\begin{frontmatter}
\title{Applying Machine Learning in Predicting Medication Treatment
Outcomes for Opioid Use Disorder}
\author[1]{Raymond R. Balise%
\corref{cor1}%
}
 \ead{balise@miami.edu} 
\author[1]{Kyle Grealis%
%
}

\author[2]{Laura Brandt%
%
}

\author[3]{Sean X. Luo%
%
}

\author[2]{Gabriel Odom%
%
}

\author[1]{Guertson Jean-Baptiste%
%
}

\author[1]{Daniel J. Feaster%
%
}


\affiliation[1]{organization={University of Miami},,postcodesep={}}
\affiliation[2]{organization={Florida International
University},,postcodesep={}}
\affiliation[3]{organization={Columbia University},,postcodesep={}}

\cortext[cor1]{Corresponding author}







        





\begin{keyword}
    machine learning \sep opioid use disorder \sep treatment
outcomes \sep medication treatment \sep 
    prediction modeling
\end{keyword}
\end{frontmatter}
    

\section{Abstract}\label{abstract}

\subsection{Background}\label{background}

While medication for opioid use disorder (MOUD) is effective for a
significant proportion of patients, many return to using opioids during
treatment. Understanding which factors lead to successful treatment is
important for the development of implementation approaches that can
improve outcomes. This manuscript and its accompanying website provide
an applied introduction to interpretable machine learning for clinical
investigators interested in predicting treatment response for people
using MOUD.

\subsection{Methods}\label{methods}

Machine learning algorithms (KNN, logistic regression with and without
regularization, MARS, Support Vector Machines, CART, Random Forest,
BART, Boosted Trees, Neural Network) were used to predict failure of
treatment in a collection of 2,478 individuals who had participated in
the three largest pragmatic, clinical trials of MOUD.

\subsection{Results}\label{results}

All models produced ROC AUC estimates in the range of 0.62 to 0.67 using
cross-validation data and the optimal model, random forest, achieved
0.65 using testing data. Predictive features, such as age, IV drug use
days, study medication, and study site, were nearly universally
identified by the algorithms. Various aspects of smoking were also
identified by most algorithms. Details from timeline follow back were
identified only by algorithms that detect complex non-linear trends. One
algorithm, BART, performed well while devaluing all treatment-specific
details.

\subsection{Conclusion}\label{conclusion}

After explaining how to apply, compare, and contrast various ML
workflows, we show that while overall modeling performance is similar
across the models built, the use of different algorithms identifies
different sets of predictive features. Some features have not been
previously identified as important for predicting treatment outcomes. To
allow further exploration of these results, a companion website provides
clinical investigators a gentle introduction to the concepts and
implementations used here. That site also provides a detailed annotated
blueprint to fully replicate our work.

\section{Introduction}\label{introduction}

Opioid use disorder (OUD) is associated with significant morbidity and
mortality \citep{hser2017, products2024, taylor2022}. Effective
treatments are available but underutilized \citep{cantor2024}. Further,
while medication for opioid use disorder (MOUD) is effective for a
significant proportion of patients, many return to using opioids during
treatment. Understanding which factors lead to successful treatment,
defined here by abstinence from nonprescribed opioids during a treatment
or observation period and beyond, is important for the development of
implementation approaches that can improve outcomes
\citep{hser1997, gustin2015, bell2020, leshner_medications_2019, wakeman2020}.
A 2024 review by Mattos and colleagues describes the application of
state of the art predictive, machine learning (ML), models to quantify
factors associated with MOUD adherence and return to opioid use
\citep{demattos2024a}. While many of the papers which they review suffer
from methodological flaws or use devices that are not feasible during
routine care (i.e., fMRI and EEG analyses) the remaining demonstrate
impressive predictive performance
\citep{burgess-hull2022, baucum2023, eddie2024}. More recently,
investigators presented one ML algorithm, logistic LASSO, as a possible
system for predicting treatment success for individuals initiating MOUD
\citep{Luo2023}. ML models can not only predict who will respond better
to MOUD, but also uncover important and interpretable reasons of
treatment success/failure through the applications of a body of
state-of-the-art techniques in explainable artificial intelligence
\citep{molnar2022a, biecek2021a}. While the recently published model by
Luo et al.~(2023) is conceptually simple and emphasizes ease of clinical
application, there is a case to be made for describing more than a
``best'' possible predictive model. Computer scientists have shown that
there is no best/optimal ML method. Furthermore, it may be that more
complex models can clarify which predictors are the most important.
Various ML algorithms highlight different patterns within patient
behavior and treatment. Thus, examining multiple predictive models
affords clinical insights beyond optimal prediction
\citep{nofree2024, biecek2021a}. That is, different modeling methods use
different kinds of information (i.e., linear, nonlinear, nonparametric
trends) so they can utilize different kinds of complexity (i.e.,
traditional statistical interactions vs.~the importance of different
variables in subgroups). Here, we expand on the ML method in
\citep{Luo2023} and explore the performance of nine algorithms useful to
predict treatment success/failure for OUD with MOUD, and we assess data
features for variable importance with the aim of detecting cross
methodological consistencies.

Traditional statistical modeling approaches to predicting treatment
success or failure have focused on finding statistically significant
variables that are associated with treatment outcomes. Researchers have
built models that are optimized to give the best possible estimates of
the influence of predictors on the full set of data collected in a study
sample. Typically, these effects are reported as beta parameter
estimates or as odds/risk ratios with confidence limits. While ML
practitioners care about these estimates for the predictor variables,
they typically optimize their results based on the ability to make
predictions in a \emph{new} sample. While traditional statistical
workflows involve variable transformations, ML workflows expand and
systematize this variable ``feature engineering'' process. Models are
built and optimized on a subset of the data, called the \emph{training}
data, to give the best predictions on new data after the variables have
been extensively processed and turned into ``features.'' That new data,
called the \emph{testing} subset, is a random sample of the original
data, which was set aside at the start of the modeling process. The
final review of model performance on testing data is critical because ML
methods can ``overfit'' the data. That is, if the predictions of a ML
model are applied to the original training data, they will likely give
overly optimistic estimates of performance because they will detect and
use subtle patterns in the training data that will not replicate in a
new sample. Optimal performance can be defined in many ways, but popular
options include accuracy of prediction or a combination of the true
positive and false positive rates (i.e., area under a Receiver Operating
Characteristic (ROC) curve, which is a plot that shows the true positive
rate against the false positive rate across all classification
thresholds) in a new sample
\citep{james2021a, fawcett2006, bradley1997}. To achieve optimal
predictive performance, many different ML algorithms are tried and one
is selected as the winner. Then that ``winning'' algorithm is applied to
the ``testing'' results. We do that here, but also for illustration, we
provide information showing how the models performed on the training
data and to illustrate how the other models would have fared if they had
been selected instead as the ``winner''.

Beyond making predictions, many ML methods afford the opportunity to
identify important variables that are missed using traditional
statistical methods. Unlike traditional statistical methods that focus
on model method-specific metrics to determine whether a variable is
important (e.g., the size of odds/risk ratios and confidence intervals
in logistic regression), in the ML framework, the importance of
variables can be assessed differently. There are model-agnostic methods
that can be used to determine if a variable is important. For example,
permutation-based variable importance, which can be calculated for most
ML methods, looks at the degradation in model performance if the values
of a variable are shuffled and randomly assigned to new study
participants. If the accuracy or ROC AUC (Receiver Operating
Characteristic Area Under the Curve) does not change when the values of
a variable are shuffled, then that variable was not important to the
model. However, if the ROC AUC decreases after a variable is shuffled,
then the amount of change in the ROC AUC is indicative of the variable's
importance in the model. Here, we provide information on the relative
importance of the variables selected by most of the different ML engines
and comment on the insights that can be gained by looking at the
similarities and differences across the various machines.

The aim of this manuscript is to provide an illustration for clinical
investigators of important tasks in ML applied to MOUD. Given that most
clinical investigators were not trained in the application of ML
methods, there is also an extensive companion website
\citep{balise_supplement_2025} which provides a gentle introduction to
the methods used, as well as the model tuning details. This supplement
allows clinical investigators to i) explore the relative value of
different techniques when predicting return to use of opioids (and
potentially any number of other treatment outcomes of interest), and ii)
uncover predictors identified consistently by different algorithms, as
well as important predictors that can only be found using specific
algorithms. Further, there is a brief technical reference
\citep{balise_technical_2025} that provides modeling details for ML
practitioners. These tools build a computational framework for addiction
researchers to examine and explain future predictive models in substance
use disorder research beyond MOUD.

\section{Methods}\label{methods-1}

\subsection{Subjects}\label{subjects}

A total of 2,478 individuals, all who participated in one of three
studies supported by the National Drug Abuse Treatment Clinical Trials
Network (CTN) and who had any self-reported drug use information before
and after randomization, were included in the analyses
\citep{saxon2013, weiss2011, lee2018, R-public.ctn0094data}. These
studies represent the three largest pragmatic, clinical trials to date
of MOUD: Starting Treatment with Agonist Replacement Therapies (START,
CTN-0027), Prescription Opioid Addiction Treatment Study (POATS,
CTN-0030), and Extended-Release Naltrexone (XR-NTX) vs.~Buprenorphine
(BUP-NX) for Opioid Treatment (X:BOT, CTN-0051). These trials included
treatments with methadone, buprenorphine, or extended-release injection
naltrexone (XR-NTX). Details on the study samples and the harmonization
process for the data are available in the technical reference
\citep{balise_technical_2025}, web supplement
\citep{balise_supplement_2025}, and published elsewhere
\citep{raymondbalise2024}.

\subsection{Predictors of Failure of
Treatment}\label{predictors-of-failure-of-treatment}

Previous research has suggested that Social Determinants of Health
(SDOH), criminal involvement, reliance on public assistance, limited
education, unemployment and homelessness, as well as clinical
characteristics including baseline pain, comorbid psychiatric issues and
the use of alcohol and other drugs, influence the probability of
treatment retention and---ultimately---success
\citep{hser1997, biondi2022, pretreat1981, comptoniii2003, mclellan1983}.
Many of these features, along with information on treatment, can be
found in data sets from the CTN0094 data harmonization project used
herein. This data is available within a software package for the R
programming language, \texttt{public.ctn0094data}
\citep{public.ctn0094data}. While most of the key features that have
been identified to predict treatment outcomes are available in
\texttt{public.ctn0094data}, a few important features are not available,
such as criminal involvement, previous treatment attempts (number, type,
duration and outcomes), and total years of drug use, as well as details
about access to non-study services and needs, particularly
transportation, child care, vocational services, and housing services.

A total of 46 variables, from domains including demographics, comorbid
psychiatric/medical conditions, smoking history, lifetime drug use
history, drug use as both yes/no indicators and amount in the month
before treatment, and treatment details (inpatient vs.~outpatient and
treatment medication) were used as predictors (see technical reference
\href{https://github.com/CTN-0094/ml_paper_2025/technical_reference.pdf}{Tables
1 and 2} and the web supplement
\href{https://ctn-0094.github.io/ml_paper_2025/supplement.html\#table1}{Table
1} and
\href{https://ctn-0094.github.io/ml_paper_2025/supplement.html\#table2}{Table
2}) \citep{balise_technical_2025, balise_supplement_2025}. As is
customary when using ML, several preprocessing recipes were evaluated
\citep{kuhn2022}. Here, we built preprocessing recipes that covered all
the necessary steps (e.g., normalization, dummy coding etc.), for all
the ML methods we used \citep{silge}. We also experimented with other
techniques to reduce the complexity of the predictors (i.e., including
principal components analysis as a preprocessing step). Various
combinations of these steps were applied before trying the various ML
methods. As can be seen in the web supplement
\citep{balise_supplement_2025}, the various recipes resulted in models
being built using between 15 and 103 individual features. Ultimately,
the recipe that resulted in the best ROC AUC on the training data was
used. In order, those steps were: remove near zero variance variables,
convert text strings to categorical factors, impute missing values using
a K nearest neighbors algorithm, dummy-code categorical variables,
collapse rare categories, remove highly correlated predictors and then
normalize numeric variables.

\subsection{Outcome}\label{outcome}

Many methods have been proposed to evaluate the success or failure of
treatment for OUD. A recent article by Brandt et al.~showed that the
model metrics cluster into three different groups based on how they
handle missing data and the length of the time frame used to judge
success \citep{brandt2024}. Here, we use one of the metrics from their
so-called `Goldilocks' cluster where the outcomes are not too lenient,
like the metrics that ignore missing urine drug screening, and not too
strict, like the metrics that look at only a single or a couple of UDS
values. Specifically, we use the definition of failure of treatment
reported by Lee et al., which looks for four consecutive weeks with a
UDS positive for opioids starting 21 days post-randomization
\citep{lee2018}. See the web supplement \citep{balise_supplement_2025}
for an example using this code.

While our outcome here followed an abstinence-based definition, we
recognize the growing emphasis on alternative endpoints --- such as
reductions in use, craving, anxiety, and quality of life --- that are
identified as meaningful by people with substance use disorders. To
support these perspectives, we have developed open-source tools (e.g.,
public.ctn0094extra \citep{public.ctn0094extra} and CTNote
\citep{CTNote}) that compute over 50 published endpoints, laying the
groundwork for future analyses that move beyond abstinence to capture
broader dimensions of treatment success.

\subsection{Modeling}\label{modeling}

Analyses were conducted with R version 4.5.0 with the
\texttt{tidyverse}, \texttt{rUM}, \texttt{table1} and \texttt{flextable}
packages used to preprocess and summarize data
\citep{R-base, R-tidyverse, tidyverse2019, R-rUM, R-table1}. We used the
tidymodels ecosystem for modeling and DALEX and DALEXtra packages for
interpretable ML processing \citep{R-tidymodels, DALEX, DALEXtra}.
Techniques include regression-based (i.e., LASSO, logistic, MARS),
tree-based (i.e., BART, Boosted Trees, CART, Random Forest) and neural
network methods. Tree-based and neural network methods are particularly
adept at detecting complex treatment medication-specific effects that
could be missed by traditional regression approaches. See the web
supplement \citep{balise_supplement_2025} for details.

The full analysis dataset was split using a stratification algorithm
that made sure that the training data (3/4 of the data, N = 1,858) and
the testing data (1/4 the data, N = 620) had the same percentage of
people for whom treatment failed. For model tuning, five-fold cross
validation was used. See the web supplement
\citep{balise_supplement_2025} for more details.

Given the historical popularity of ROC AUC, its relatively easy
interpretation, and the fact that there was not a major imbalance in the
proportion of people for whom treatment was unsuccessful as per the
metric introduced above (72\% with unsuccessful vs.~28\% with successful
treatment outcomes), all models were tuned to find the optimal ROC AUC
\citep{weiss2013}. If you are not familiar with how to interpret a ROC
AUC value, assume you have two people, for one of whom treatment will
work and for one of whom it will not. The ROC AUC value gives the
probability of correctly guessing for which one of whom the treatment
will fail. While the definition of a clinically meaningful improvement
in a model is largely arbitrary, \emph{a priori} we defined a meaningful
difference as a 5\% improvement, when comparing any two tuned models, in
the probability of correctly guessing for whom treatment would fail.
Results from all models were examined using permutation-based Variable
Importance Plot (VIP) methods and confusion matrices (see
\citep{molnar2022b} and the web supplement
\citep{balise_supplement_2025}). The details on model tuning, including
a definition of the terms for those unfamiliar, can be found in the web
supplement \citep{balise_supplement_2025}.

\subsection{Evaluation Metrics}\label{evaluation-metrics}

Consider that there are many possible algorithms that can be used to
make predictions and only a single test dataset that can be used to
estimate a model's performance on new data. Evaluating the performance
of every algorithm on the test data would likely result in an overly
optimistic estimate of model performance. This would happen because with
\emph{many} different models, we would eventually get ``lucky'' and
obtain particularly good performance simply by chance. Because of this,
it is customary and appropriate to optimize the results for each
algorithm and then select a ``winner'' based on a metric like accuracy
or ROC AUC \emph{using the training data alone.} The algorithm that
performs the best on the preselected metric, say ROC AUC, is then
applied to the test data and the ROC AUC from the ``testing'' data is
reported as the estimate of future performance on unseen new data.

\section{Results}\label{results-1}

Here, because we are interested in explaining the ML workflow, we have
taken a few extra steps. Table~\ref{tbl-one} presents the model
performance, as assessed by ROC AUC, for each algorithm. The ``Cross
Validation'' results are typically used to pick the ``winning''
algorithm from the training data. It is an estimate based on repeatedly
subsampling the training data. This process is explained in detail in
the web supplement \citep{balise_supplement_2025}. The top row of
Table~\ref{tbl-one} shows that the best performing algorithm was the
random forest algorithm. The ``Full Training Dataset'' column shows the
result when the model is applied to the full training dataset to
estimate future performance. This is what has historically been done
with logistic models. Unsurprisingly, because the model is predicting
the data it was built on, the Full Training Dataset results are
optimistic. When the random forest model or KNN is applied to the full
training data, you see that the model performance is extremely good and
overly optimistic because the model uses the idiosyncratic features in
the data as well as the true signal.

The random forest algorithm has the largest cross-validation ROC AUC
(0.67), so it could be called the ``winner.'' In typical real work
applications, this would be the only model applied to the testing data.
The results from such a ML model would be that: using the ``winning''
random forest model and the data gathered before treatment, from a pair
of people, for one of whom treatment will succeed while it fails for the
other, we have about a 67\% chance of correctly guessing for which one
it will fail. While this is the best model, that value is not ``good.''
Many experts including Hosmer, Lemeshow, and Sturdivant
\citep{hosmer2013}, who say ``there is no `magic' number, only general
guidelines'', consider a value of 0.7 ``acceptable discrimination''.
This recapitulates what clinical investigators know: predicting failure
of treatment, even when given all traditionally gathered variables, is
extraordinarily difficult and the predictions should be treated with
great caution.

The importance of using metrics such as ROC AUC to assess predictive
performance is particularly notable when contrasted with the insights
provided by a traditional, p-value-based interpretation of a logistic
regression model. Here, the traditional logistic regression model
identifies five features with p-value \textless{} .001 and twenty-two
with p-values \textless{} .05. Clearly, there are many important
features that drive the odds of failure of treatment, but the overall
impact of these features is not large because that model has a ROC AUC
of ``only'' 0.63 when applied to the testing data.

While in real world applications the test data is only used once, here
(again for illustrative purposes), we applied each of the methods, with
their optimal tuning, to the testing data. We see that the cross
validation estimates of future performance match nicely with the
behavior on the testing data, in other words, the cross-validation
methods used allowed us to pick up on the true signal without
``overfitting'' our models to pick up on random noise in the training
data.

Figure~\ref{fig-one} shows the true vs.~predicted results for the people
in the test data as a confusion matrix, and the predicted probability of
failure for those who did and did not return to opioid use. Both the
density and the confusion matrix show the relatively high chance of
predicting return to use. Ideally, the probability plot would show a
bathtub shape, where the people who experienced failure of treatments
were all close to the right side of the plot and the successes were
close to the left. Here, the model is rarely assigning a low probability
of failure of treatment.

The ML methods described above selected different numbers of features as
being important to predict failure of treatment. While the kNN algorithm
uses, by design, every feature, and determining the number of features
used by other algorithms like NN and SVM is extremely difficult for
technical reasons, there are noteworthy differences in the number of
features the different algorithms use (MARS N = 15, CART N = 28, boosted
trees N = 99, LASSO N = 96, logistic regression N = 99, random forest N
= 102). At the extremes, the MARS model identified 15 distinct variables
as important, and the random forest chose 102 different binary splits
from 46 variables. For example, the random forest model made eight
splits based on the scores of 1 through 8 on the Fagerstrom Test for
Nicotine Dependence. Interestingly, across all the methods, a total of
105 different features were identified as being important predictors at
least once.

\begin{longtable}[]{@{}
  >{\raggedright\arraybackslash}p{(\linewidth - 6\tabcolsep) * \real{0.2143}}
  >{\raggedleft\arraybackslash}p{(\linewidth - 6\tabcolsep) * \real{0.2429}}
  >{\raggedleft\arraybackslash}p{(\linewidth - 6\tabcolsep) * \real{0.3143}}
  >{\raggedleft\arraybackslash}p{(\linewidth - 6\tabcolsep) * \real{0.2286}}@{}}

\caption{\label{tbl-one}Overall model performance, measured using area
under the ROC curve, when predicting failure of treatment in a training
data set of 1,858 people seeking care for Opioid Use Disorder.}

\tabularnewline

\toprule\noalign{}
\begin{minipage}[b]{\linewidth}\raggedright
Model
\end{minipage} & \begin{minipage}[b]{\linewidth}\raggedleft
Cross Validation
\end{minipage} & \begin{minipage}[b]{\linewidth}\raggedleft
Full Training Dataset
\end{minipage} & \begin{minipage}[b]{\linewidth}\raggedleft
Testing Dataset
\end{minipage} \\
\midrule\noalign{}
\endhead
\bottomrule\noalign{}
\endlastfoot
Random Forest & 0.674 & 0.889 & 0.647 \\
Boosted Trees & 0.672 & 0.839 & 0.645 \\
BART & 0.667 & 0.748 & 0.623 \\
LASSO & 0.666 & 0.720 & 0.617 \\
Neural Net & 0.658 & 0.712 & 0.636 \\
KNN & 0.648 & 0.931 & 0.630 \\
Logistic LASSO & 0.648 & 0.742 & 0.618 \\
Support Vector & 0.647 & 0.710 & 0.597 \\
MARS & 0.641 & 0.715 & 0.615 \\
CART & 0.623 & 0.670 & 0.574 \\
Logistic & NA & 0.744 & 0.626 \\

\end{longtable}

\begin{figure}

\centering{

\pandocbounded{\includegraphics[keepaspectratio]{paper_files/figure-pdf/fig-one-1.pdf}}

}

\caption{\label{fig-one}Categorical predictions (using a threshold of
0.5) and predicted probabilities of failure of treatment for individuals
in the testing data set (N = 620) when applying the random forest
model.}

\end{figure}%

As discussed in the web supplement \citep{balise_supplement_2025},
permutation-based VIP measures show the relative importance of different
predictors. Figure~\ref{fig-two} shows an estimate of how much our best
performing model, a random forest, would be hurt by excluding different
variables. For example, the overall ROC AUC score would decrease by
somewhere between 0.75\% and 1\% if the \emph{age} was not included.

Table 2 shows that some predictors, like \emph{Age}, \emph{IV drug use
days}, \emph{Study Medication} and \emph{Study Site} were nearly
universally selected. Other recurring predictors include: \emph{Detox
Days} and \emph{Fagerstom} were identified by 7 algorithms. While many
of the most important predictors are categorical, the logistic
regression, LASSO and MARS methods were able to better detect linear
trends for continuous variables. That is, they captured trends in
\emph{age} and the \emph{number of days using drugs in the 28 days
before treatment} as well as \emph{IV drug use days}. Nearly every
method identified smoking (i.e., \emph{person who smokes}, \emph{number
of packs smoked} or the \emph{Fagerstrom score}) as being related to
failure of treatment. Interestingly, BART, which used many different
predictors compared to the other methods, achieved similar model
performance using many different variables. The interpretation of the
impact of each predictor requires finesse beyond a simple statement
about overall increased or decreased risk of failure
\citep{biecek_explanatory_2021}.

\begin{landscape}
\vspace*{\fill}

\begin{figure}

\centering{

\pandocbounded{\includegraphics[keepaspectratio]{paper_files/figure-pdf/fig-two-1.pdf}}

}

\caption{\label{fig-two}Variable Importance Plot (VIP) for Random Forest
Model.}

\end{figure}%

\vspace*{\fill}
\end{landscape}

\begin{landscape}

\begin{table}

\caption{\label{tbl-two}Top 10 variables identified by permutation
testing for predicting failure of treatment for 2,478 people seeking
care for Opioid Use Disorder.}

\centering{

\fontsize{12.0pt}{14.4pt}\selectfont
\begin{tabular*}{\linewidth}{@{\extracolsep{\fill}}rllll}
\toprule
Importance & BART & Boosted Trees & CART & LASSO \\ 
\midrule\addlinespace[2.5pt]
1 & Brain Damage DX & Age & Detox Days & Study Site \\ 
2 & Marital Status & Study Medication & Study Medication & Age \\ 
3 & Used Stimulants & IV Drug Days & Age & Study Medication \\ 
4 & In/Outpatient & Study Site & IV Drug Days & IV Drug Days \\ 
5 & Cocaine Days & Smoking Packs & Job & Study Number \\ 
6 & Homeless & Detox Days & Study Site & Detox Days \\ 
7 & Depression & Opioids Days & Cocaine Days & Fagerstrom \\ 
8 & Schizophrenia & TLFB Days of Use & Smoking Packs & Alcohol DX \\ 
9 & Used Cocaine & Fagerstrom & Alcohol DX & Used Opioids \\ 
10 & Alcohol DX & TLFB Number of Drugs & Person Who Smokes & Job \\ 
\bottomrule
\end{tabular*}
\begin{minipage}{\linewidth}
{\textcolor[HTML]{FFFFFF}{x}}\\
\end{minipage}

\fontsize{12.0pt}{14.4pt}\selectfont
\begin{tabular*}{\linewidth}{@{\extracolsep{\fill}}rlllll}
\toprule
Importance & Logistic & Logistic via LASSO & MARS & Neural Net & Random Forest \\ 
\midrule\addlinespace[2.5pt]
1 & Study Site & Study Site & Detox Days & Study Site & Age \\ 
2 & In/Outpatient & Age & Study Number & Fagerstrom & Smoking Packs \\ 
3 & Age & In/Outpatient & Study Site & Study Medication & Study Site \\ 
4 & Smoking Packs & Study Medication & Age & Age & Study Medication \\ 
5 & Study Medication & IV Drug Days & Opioids Days & Used Heroin & Detox Days \\ 
6 & Study Number & Fagerstrom & Study Medication & Job & Fagerstrom \\ 
7 & Person Who Smokes & Job & IV Drug Days & Smoking Packs & IV Drug Days \\ 
8 & IV Drug Days & Detox Days & Fagerstrom & IV Drug Days & TLFB Days of Use \\ 
9 & Fagerstrom & Stimulant Days & Alcohol DX & Anxiety & TLFB Number of Drugs \\ 
10 & Detox Days & Alcohol DX & Race & Withdrawl & Withdrawl \\ 
\bottomrule
\end{tabular*}

}

\end{table}%

\end{landscape}

\section{Discussion}\label{discussion}

The goal of this manuscript with its web supplement
\citep{balise_supplement_2025} was to provide a self-contained tutorial
on modern ML methods and to provide an example of how these methods can
be used to predict treatment response in people with OUD. Despite using
these state-of-the-art methods, our results show that our predictions
for treatment success or failure are far from ideal. Using these models
and the data gathered before treatment, for a pair of people, for one of
whom treatment will succeed while it will fail for the other, we have
about a 65\% chance of guessing for which one it will fail. While the
models afford similar overall performance, we can learn by
exploring/contrasting their findings.

Different modeling methods detected some common threads. It is
unsurprising to see that IV drug use, a risk factor that has been tied
to return to opioid use, was selected as an important predictor by all
but one method \citep{chalana2016}. Polysubstance drug use, in
particular the use of cigarettes, was nearly universally identified as a
predictor of failure. Several tree-based methods, which are ideal for
picking up non-linear patterns, stressed the importance of
length-of-stay in a detoxification unit before study drug
administration. To better understand this impact, consider that our
models also included details on the amount of illicit drug use before
the intervention. The relatively complex and best performing models
(i.e., random forests and boosted trees) identified days of drug use
before care, as well as days in detox. These findings compliment other
analyses that use latent class methods (a technique that does not
integrate well with a ML framework), which have identified drug use
trajectories in the month before treatment \citep{pan2022}. Perhaps,
drug use patterns before the start of the trial and time in detox are
reflecting people's ability to reduce drug use before MOUD initiation.

It is interesting that one of the models, BART, uniquely identified many
variables as predictors. This is an example of the ML ``Rashomon
Effect'' and it suggests the need for further work to explore whether
several equally viable explanations of treatment outcomes can be stacked
using ensemble ML methods
\citep{breiman1996, wolpert1992, paes2023, rudin}. While it will come at
a cost of interpretability, we strongly suspect that we will be able to
improve our predictive power by using ML techniques that combine
algorithms.

Furthermore, the fact that study site was nearly universally identified
is noteworthy. As discussed elsewhere, the sites were aggregated to
ensure anonymity \citep{raymondbalise2024}. Thus, it is possible that
the site effects are more important than we document here. While all
subjects were participating in highly structured clinical trials, the
study site was an extremely important prognosticator indicating that the
local context is important in predicting individual outcomes. However,
it is unknown if this feature is capturing details within the clinic,
such as the similarity in age, sex, race, and lived experience between
the patients and care providers or general positive or negative
attitudes/interactions or issues related to ease of access or the
broader social milieu. In theory, details on care providers could be
added to future models. While it is technically simple to integrate data
from the census to describe the neighborhoods where care was provided,
there are thorny privacy and ethical issues. Sensitive details could be
used to easily reverse engineer the individual clinics which, in turn,
could lead to an increased risk of identifying individuals. Methods for
dealing with these site-specific effects (which would be considered
random effects in traditional statistical models) are only now becoming
more relevant as interpretable ML methods like the VIP values become
more popular. As ML methods move beyond their infancy, we expect new
techniques will be developed to assess or remove the impact of clinic
sites on predictions. However, rather than controlling for or removing
those effects, we should learn what drives them.

While all methods were evaluated using ROC AUC, it is important to
remember that there are many metrics that could be used to evaluate a
model \citep{sokolova2009, opitz2024}. The choice of metrics is an
active area of research. Different metrics emphasize different aspects
of model performance: Matthews Correlation Coefficient balances
true/false positive and negative rates while accounting for class sizes,
Log Loss penalizes incorrect predictions more heavily, Precision-Recall
AUC focuses on the positive class, and Cohen's Kappa emphasizes
performance on difficult cases where chance guessing performs well.
Several such metrics are described and calculated in the web supplement
\citep{balise_supplement_2025}.

Depending on research and clinical goals, optimizing for sensitivity,
specificity, or other metrics may be more valuable than ROC AUC. Beyond
classification accuracy, it is important to examine the underlying
probabilities that drive predictions. Rather than focusing solely on
whether the model correctly predicted treatment failure, researchers
should examine the predicted probabilities for individuals, as shown in
Figure 1B. These probabilities reveal whether a model is highly
confident or uncertain in its estimates for specific individuals.

In this example, the model was highly confident in predicting treatment
failure for many people who actually succeeded in treatment (i.e., did
not return to opioid use during treatment).

The number of sexual partners was measured, to assess HIV risk, in
CTN-0027 and CTN-0030, but the days in detox was measured only in
CTN-0051. Consequently, these variables pose particular challenges for
modelling. Unfortunately, the sexual partners variable caused the
permutation-based VIP methods to fail, so we could not assess its
relative importance here. We feared that having a value on a variable
could act as a simple proxy for being in a particular trial. Because of
this concern, we conducted numerous sensitivity analyses using different
subsets and imputation/simulation methods. These techniques did diminish
the relative importance of the predictors, but they remained
influential.

Our models would certainly benefit from information that has been
identified as relevant to the prediction of treatment success or failure
in other studies
\citep{hser1997, biondi2022, pretreat1981, mclellan_increased_1983, comptoniii2003, mclellan1983}.
We used all data available on tobacco and alcohol use but because the
data were not uniformly assessed across the trials, we could not include
as much detail on consumption as we wanted. Similarly, detailed
information on the route of drug use (e.g., oral vs.~nasal use of opioid
pills) could not be used. We believe that contextual factors such as
transportation barriers, access to childcare, etc. impact the
probability of attending and ultimately succeeding in care. Further,
information on previous treatment attempts would be valuable, and
genetic, genomic, and other -omic data will likely ultimately prove
valuable in predicting treatment response and matching people to the
best possible medications.

In conclusion, ML methods are useful for making predictions of how
individual patients will respond to treatment; however, current data
provide only modest veracity in these predictions. The utility of
current ML models is limited by the quality, breadth, and depth of the
data collected in MOUD clinical trials. We hope that future studies will
systematically and richly assess features related to the
provider-patient relationship, provider beliefs about MOUD, patient
attitudes towards MOUD, prior treatment attempts, socioeconomic and
demographic characteristics, polysubstance use, risky sexual behaviors,
and the many other potentially important features highlighted in the
discussion and limitations section. To this end, the code provided with
the web supplement \citep{balise_supplement_2025} allows anyone who is
familiar with the R programming language to extend our work and explore
other theories and predictive features. Such work will likely allow us
to improve the quality of our predictions and identify missing key
features for predicting treatment success.


\renewcommand\refname{References}
\bibliography{references.bib,packages.bib}



\end{document}
